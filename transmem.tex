\documentclass{acm_proc_article-sp}
\usepackage{graphicx}
\usepackage{amssymb}
\bibliographystyle{plain}

\title{Virtualized Transactional Memory}
\author{Zachary Bischof \\
	Northwestern University\\
	Evanston, IL, USA\\
	zbischof@u.northwestern.edu
	\and 
	Marcel Flores \\
	Northwestern University\\
	Evanston, IL, USA\\
	marcel-flores@u.northwestern.edu
	\and
	Maciej Swiech \\
	Northwestern University\\
	Evanston, IL, USA\\
	maciejswiech2007@u.northwestern.edu
	}

\begin{document}
\maketitle

\begin{abstract}
Transactional memory has long been considered as an alternative to locks for
implementing parallel algorithms. In particular, it offers an optimistic
approach to parallelization, and attempts to perform an operation assuming no
contention will occur.

Despite the potential gains of transactional memory, it is only in the 
not-yet-released Haswell architecture from Intel that it has made an appearance.
Palacios has presented us with a unique opportunity to emulate this Intel
implementation in software.

In this paper, we present a basic implementation of transactional memory, which
emulates the Haswell instructions, successfully performing memory transactions.
\end{abstract}

\section{Introduction}
Transactional Memory is pretty cool stuff. We decided to implement it.

\section{Background}
Transactional memory has long been proposed as a solution to the complexity 
that arises from attempting to use locks in real world implementations of 
parallel algorithms. We now discuss the basic model of a transaction.

\subsection{Transactions}
Transactions are a sequence of memory operations that are performed by a 
process such that the operations appear to happen serially and atomically
~\cite{Herlihy:1993:TMA:173682.165164}. When a transaction is finished, it 
either commits its changes to main memory, or it aborts the transaction
and throws away its changes. In particular, when a transaction begins, it
keeps track of all memory which is read and written during the transaction.
If another process reads or writes this memory, the transaction is aborted.
Upon an abort, a transaction may decide to attempt to run again, or revert
to traditional locking mechanisms.

In addition to memory conflicts, which violate the atomic requirement, 
transactions may also be aborted in the event of interrupts, context switches,
or other forced changes in the control flow, as they would violate the
requirement that the transaction happen entirely serially. 

Transactions of this form can be thought of as an optimistic approach to
concurrency, contrasting with the more pessimistic approach of locks. 
Rather than explicitly preventing other processes from accessing memory, the
transaction assumes that it will probably be able to complete unencumbered. If
a conflict does occur, the transaction is able to deal with it accordingly.
Locks, on the otherhand, assume that conflicts are likely to occur, and 
directly block other processes from accessing the relevant portions of memory.
Transactions therefor stand to offer a performance gain, as forced 
serialization can be avoided in many cases.

In addition, transactional memory stands to decrease the complexity of 
multi-process applications. Programmers would no longer be weighed down by 
keeping track of which process is holding which locks, and no longer needs
to worry about deadlocks and other such situations.

\subsection{Intel Implementation}
The Intel implementation generally follows the design of 
~\cite{Herlihy:1993:TMA:173682.165164}, which a few exceptions.
Most notably, each transaction is only attempted a single time: in the event
of a failure or abort, the code immediatly runs the failsafe. Additionally, 
the transactions are checked at a granularity of cache lines, rather than
actual memory address. 

Other changes?

The Palacios virtual machine manager offers us a unique opportunity to
implement transactional memory without explicit hardware support.

\section{Palacios Implementation}
In order to implement transactional memory in Palacios, we follow Intel's
Haswell specifications, providing us with a consistent interface with which
to use transactions, as well as making future comparisons to direct hardware
implementations easier. 


\section{Design Challenges}

\subsection{Current Assumptions}

\section{Future Work}

\section{Conclusions}

\bibliography{transmem}

\end{document}
